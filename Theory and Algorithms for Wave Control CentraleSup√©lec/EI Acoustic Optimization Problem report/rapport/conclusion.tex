\section{Conclusion}

Cette étude nous a permis de nous familiariser encore plus avec le concept de la réduction de la pollution acoustique.\\ \\
Pour la partie numérique, nous arrivons à une solution satisfaisante puisque qu'après l'étude de nombreux paramètres, nous avons avec, une quantité de matériaux raisonnable (clairement pas maximale), réussi à réduire considérablement le bruit au sein du réacteur d'avion. \\ \\
Pour la partie théorique, nous avons réussi à justifier la validité du problème sous-jacent (son caractère bien posé, ainsi que les propriétés de l'énergie acoustique de la solution) ce qui permet de justifier les méthodes utilisées dans la partie théorique pour optimiser réduction de l'énergie acoustique au sein du réacteur.
