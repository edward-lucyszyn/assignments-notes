\section{Blackbody Radiation and Wien's Displacement Law}

\noindent \textbf{1)} At the limit of small wavelengths, show that the Planck distribution has a maximum at the wavelength $\lambda_{\max}$ such that $\lambda_{\max} T=C$, where the value of $C$ depends on Planck's constant $h$.\\

\begin{breakbox}
    \noindent At the limit of small wavelengths, the Planck distribution (or Planck's law) is given by:
    $$dU \approx \frac{8\pi h c}{\lambda^5}e^{-hc/(\lambda k_B T)}d\lambda.$$
    Thus, we obtain $\displaystyle \frac{dU}{d\lambda} \approx 0$ when:
    $$\boxed{\lambda_{max} T \approx \frac{hc}{5k_B}.}$$
\end{breakbox}

\medskip

\noindent \textbf{2)} Use the experimental value $C=2.9 \mathrm{~mm} . \mathrm{K}$ to deduce an approximate value of $h$.\\

\begin{breakbox}
    \noindent The product of $\lambda_{max}$ and $T$ is indeed a constant $C$ which is $C = h \times 0.434 \times 10^{31}$ m.K.
    This result can be compared to the experimental value $C \approx 2.9 \times 10^{-3}$ m.K to find an approximate value of $h \approx 6.68 \times 10^{-34}$ J.s.
\end{breakbox}
