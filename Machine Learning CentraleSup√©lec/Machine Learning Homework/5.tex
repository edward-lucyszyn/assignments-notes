\section{Exercise V: Nearest Neighbours and kd-Trees}

(a) Here is the code.

\begin{Python}
    #%%
    ## Importing libraries
    
    import numpy as np
    
    #%%
    ## Exercice V: Nearest Neighbours and kd-Trees
    
    # Given coordinates
    coordinates = [
        0.28, 0.13, 0.65, 0.31, 0.61, 0.56, 0.75, 0.15,
        0.71, 0.12, 0.97, 0.28, 0.38, 0.97, 0.97, 0.15,
        0.03, 0.13, 0.96, 0.44, 1.00, 0.97, 0.43, 0.79,
        0.47, 0.00, 0.04, 0.81, 0.10, 0.44, 0.42, 0.26,
        0.93, 0.31, 0.33, 0.57, 0.44, 0.18, 0.07, 0.44,
        0.26, 0.50, 0.64, 0.49, 0.42, 0.29, 0.06, 0.78,
        0.55, 0.65, 0.40, 0.75, 0.55, 0.33, 0.77, 0.93,
        0.90, 0.65, 0.40, 0.94, 0.29, 0.35, 0.88, 0.12,
        0.48, 0.34, 0.95, 0.47, 0.85, 0.05, 0.56, 0.15,
        0.86, 0.83, 0.36, 0.74, 0.09, 0.69, 0.47, 0.34,
        0.50, 0.98, 0.98, 0.69, 0.67, 0.05, 0.23, 0.54,
        0.37, 0.78, 0.67, 0.35, 0.52, 0.36, 0.03, 0.54,
        0.72, 0.38, 0.72, 0.22, 0.11, 0.62, 0.65, 0.22,
        0.41, 0.09, 0.23, 0.35, 0.38, 0.65, 0.33, 0.73,
        0.95, 0.76, 0.83, 0.52, 0.44, 0.32, 0.15, 0.01,
        0.31, 0.65, 0.78, 0.33, 0.85, 0.23, 0.83, 0.21,
        0.60, 0.82, 0.33, 0.75, 0.05, 0.28, 0.70, 0.32,
        0.65, 0.17, 0.56, 0.65, 0.96, 0.47, 0.55, 0.14,
        0.28, 0.68, 1.00, 0.11, 0.53, 0.84, 0.50, 0.75,
        0.88, 0.36, 0.88, 0.97, 0.72, 0.90, 0.62, 0.31,
        0.64, 0.53, 0.64, 0.33, 0.79, 0.37, 0.85, 0.14,
        0.63, 0.80, 0.95, 0.14, 0.50, 0.25, 0.06, 0.64,
        0.82, 0.75, 0.42, 0.21, 0.42, 0.98, 0.65, 0.78,
        0.87, 0.16, 0.97, 0.80, 0.63, 0.46, 0.45, 0.04,
        0.01, 0.98, 0.99, 0.14, 0.53, 0.54, 0.57, 0.47
    ]
    
    # Reshape the coordinates into a numpy matrix
    data = np.array(coordinates).reshape(-1, 4)
    
\end{Python}

\begin{Python}
    #%%
    
    def sorted_index(list):
        """Return the index of the median sorted value of a list."""
        return sorted(range(len(list)), key=lambda i: list[i])

    def choose_pivot_index(list):
        """Return the index of the pivot of a list."""
        return len(list) // 2
    
    def split_data(points, points_index, depth):
        """Split the data into two parts according to the pivot of the points."""
        # Get the dimension of the points
        dimension = points.shape[1]
    
        # Sort the points according to the median
        sorted_points = list(np.array(points_index)[sorted_index(points[points_index][:, depth % dimension])])
    
        # Get the index of the pivot
        pivot_index = choose_pivot_index(sorted_points)
    
        # Get the median point
        pivot_point = sorted_points[pivot_index]
    
        # Split the points into two parts
        left_points = sorted_points[:pivot_index]
        right_points = sorted_points[pivot_index + 1:]
    
        return pivot_point, left_points, right_points
    
    def create_kd_tree(points, points_index=-1, depth=0):
        """Create a kd-tree from a set of points."""
        if points_index == -1:
            points_index = list(range(len(points)))
    
        # If there are no points, return None
        if len(points_index) == 0:
            return None
\end{Python}

\begin{Python}
        # Split the points into two parts
        pivot_point, left_points, right_points = split_data(points, points_index, depth)
    
        # Create the node of the tree
        tree = {
            'point': pivot_point,
            'left': create_kd_tree(points, left_points, depth + 1),
            'right': create_kd_tree(points, right_points, depth + 1)
        }
    
        return tree
    
    #%%
    # Order of the tree
    
    def compute_tree_order(tree):
        """Compute the order of a binary tree with dictionaries."""
        if tree['left'] == None and tree['right'] == None:
            return [tree['point']]
        elif tree['left'] == None and tree['right'] != None:
            return [tree['point']] + compute_tree_order(tree['right'])
        elif tree['left'] != None and tree['right'] == None:
            return [tree['point']] + compute_tree_order(tree['left']) 
        else:
            return [tree['point']] + compute_tree_order(tree['left']) + compute_tree_order(tree['right'])
    
    result = list(np.array(compute_tree_order(create_kd_tree(data))) + 1)
    
    print(result)
\end{Python}
The output produced by this algorithm, utilizing the coordinates from the PDF file, is as follows (nodes are indexed from 1 to 50): 
[50, 8, 27, 44, 10, 30, 24, 7, 12, 34, 16, 1, 17, 5, 23, 28, 26, 20, 11, 38, 46, 4, 37, 49, 21, 31, 36, 14, 22, 25, 48, 18, 9, 35, 3, 32, 42, 39, 47, 6, 33, 19, 45, 15, 13, 40, 43, 2, 29, 41].\\ \\
(b) At each step of the decision tree, the algorithm selects the optimal split, which involves examining every possible splitting value for each of the $d$ features. Consequently, considering the $n$ data points and assuming linear time for sorting real numbers, the complexity of this step is $O(nd)$. This process must be repeated for every layer of the tree, occurring $h$ times. Therefore, the final complexity is indeed $O(ndh)$.
