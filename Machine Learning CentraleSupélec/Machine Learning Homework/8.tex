\section{Exercise VIII: Unsupervised Learning and Clustering}

The main distinction between the k-means and k-means++ algorithms lies in the initialization step. While the k-means algorithm randomly selects the positions of centroids, the k-means++ algorithm employs a more intelligent initialization strategy. The first centroid is chosen randomly from the set of points. Subsequently, each succeeding centroid \(c_{i+1}\) is chosen from the remaining points. The point \(x_j\) is selected with the probability:

\[
\mathbb{P}(x_j = c_{i+1}) = \frac{d(x_j, c_i)^2}{\displaystyle \sum_{k, \text{remaining points}} d(x_k, c_i)^2},
\]

where $d(x_j, c_i)$ represents the distance between centroid \(c_i\) and point \(x_j\). This initialization strategy ensures that points likely to be sufficiently distant from each other.\\ \\
The quality of the solution depends on the initialization. Choosing centroids randomly may lead the algorithm to converge to a good local minimum rather than the global minimum. Additionally, it can reduce the number of iterations required for convergence. This can be particularly useful in cases involving a large number of points, where each step takes a considerable amount of time to compute, and where local minima are more likely to exist.