\section{Introduction}

\subsection{Contexte}

Nous sommes \textsc{Nabilou}, spécialisés dans l'innovation acoustique de l'industrie aérospatiale ! Notre mission consiste à redéfinir l'avenir des voyages aériens en minimisant le bruit à l'intérieur des réacteurs d'aéronefs pour améliorer les conditions des passagers et des acteurs extérieurs.\\ \\
En effet, les moteurs d'aéronefs génèrent un niveau sonore considérable, et nous avons relevé ce défi avec détermination. Avec une équipe dédiée et une approche novatrice, nous utilisons des liners acoustiques disposés de manière optimale, assurant ainsi le meilleur compromis entre l'efficacité et le coût !\\ \\
Notre travail s'inscrit dans la continuité du cours de ST5 sur le contrôle de la pollution acoustique. Parmi nos six membres, trois d'entre eux participent à un projet de formation à la recherche sur l'équation de Helmholtz convectée à CentraleSupélec en partenariat avec l'ONERA. C'est pourquoi, au sein de ce rapport, l'équipe théorique composée de deux membres s'est chargée d'une étude plus approfondie que la normale, utilisant ladite équation d'Helmholtz convectée. Enfin, les quatre autres membres du groupe ont réalisé l'étude de la partie numérique.

\subsection{Objectifs et défis}

Afin de proposer la meilleure solution pour réduire le bruit des réacteurs, nous avons dû répondre à différents objectifs tout au long de ces cinq jours. Pour l'équipe numérique :

\begin{itemize}
    \item Choix de la forme des liners ;
    \item Choix du matériaux des liners ;
    \item Optimisation de la disposition des liners ;
    \item Optimisation multi-fréquentielle ;
    \item Recherche d'autres méthodes d'optimisation \item Optimisation de la quantité de matériaux ;
    \item Description de la solution choisie ;
    \item Comparaison de la solution choisie avec une paroi complètement absorbante.
\end{itemize}
Et pour l'équipe théorique :
\begin{itemize}
    \item S'assurer théoriquement de la validité des équations d'onde utilisées ;
    \item Montrer le caractère bien posé des équations variationnelles obtenues ;
    \item Montrer la continuité de l'énergie acoustique (qui dépend de la distribution de matériau choisie) ;
    \item Montrer l'existence de distributions optimales admissibles pour une quantité de matériau fixée (minimisant l'énergie acoustique) ;
    \item Trouver explicitement une expression de la dérivée de l'énergie acoustique satisfaisante.
\end{itemize}